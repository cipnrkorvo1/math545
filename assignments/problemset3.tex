\documentclass[]{article}
\usepackage{amsmath, amssymb, amsthm}

\usepackage{multicol}

\renewcommand{\thesubsection}{\thesection.\alph{subsection}}

\newcommand{\mrow}[3]{\left[
	\begin{array}{ccc}
		#1 & #2 & #3
	\end{array}
	\right]}
\newcommand{\mcol}[3]{\left[
	\begin{array}{c}
		#1\\
		#2\\
		#3
	\end{array}
	\right]}

\newenvironment{augmx}[1]{%
	\left[\begin{array}{@{}*{#1}{c}|c@{}}
	}{%
	\end{array}\right]
}

\newcommand{\extaugmx}[2]{
\left[\begin{matrix}
	#1
\end{matrix}\right|\left.\begin{matrix}
	#2
\end{matrix}\right]
}

\newcommand{\xlrightarrow}{\xrightarrow{\hspace{0.8cm}}}
\newcommand{\reals}{\mathbb{R}}

\newcommand{\bmat}[1]{\begin{bmatrix}#1\end{bmatrix}}

\newcommand{\letmatrix}[2]{\mathcal{M}_{#1 \times #2}(\reals)}

\newcommand{\detmat}[1]{
	\left|\hspace{0.5em}
	\begin{matrix}
		#1
	\end{matrix}
	\hspace{0.5em}\right|
}

\begin{document}
\begin{center}
	{\Large\textbf{MATH545: Problem Set 3}}\\
	\large Grey Cole\\
	February 16, 2026
\end{center}
\section{}

\begin{align*}
\detmat{1&\beta&0\\3&2&0\\1&2&1} &= (1*2*1) + (\beta * 0 * 1) + (0 * 3 * 2) - (0 * 2 * 1) - (\beta * 3 * 1) - (1 * 0 * 2)\\
&= 2 + 0 + 0 - 0 - 3\beta - 2\\
&= -3\beta
\end{align*}

Since a square matrix is noninvertible iff its determinant is zero, the given matrix is \fbox{noninvertible when $\beta = 0$.}

\section{} 
\subsection{}
No, $A$ is not necessarily invertible; A $3 \times 3$ matrix made up of all zeroes is symmetric but has no inverse.
\subsection{}
Let $A \in \letmatrix{n}{n}$ such that $A$ is symmetric and has an inverse $A^{-1}$.
\begin{align}
	A &= A^T\\
	AA^{-1} &= A^{-1}A\\
	A^TA^{-1} &= A^{-1}A^T\\
	(A^TA^{-1})^T &= A^{-1(T)}A\\
	A^{-1}A &= A^{-1(T)}A\\
	A^{-1} &= A^{-1(T)}
\end{align}
Using properties of inverse matrices and matrix transpose, we have derived \fbox{$A^{-1} = A^{-1(T)}$}, meaning \underline{$A^{-1}$ is symmetric.}

\section{}
\subsection{}
Find the inverse by performing Gauss-Jordan Elimination on the coefficient matrix augmented with the $3\times 3$ identity matrix:
\begin{align*}
	&&
	\extaugmx{-2&-2&-1\\-1&-1&0\\0&-1&2}{1&0&0\\0&1&0\\0&0&1}&\\
	R_1-2R_2 \rightarrow R_1 &\xlrightarrow&
	\extaugmx{0&0&-1\\-1&-1&0\\0&-1&2}{1&-1&0\\0&1&0\\0&0&1}&\\
	R_3 + 2R_1 \rightarrow R_3 &\xlrightarrow&
	\extaugmx{0&0&-1\\-1&-1&0\\0&-1&0}{1&-1&0\\0&1&0\\2&-4&1}&\\
	R_2-R_3 \rightarrow R_2 &\xlrightarrow&
	\extaugmx{0&0&-1\\-1&0&0\\0&-1&0}{1&-2&0\\-2&5&-1\\2&-4&1}&\\
	\begin{array}{c}
		-R_2\rightarrow R_1\\
		-R_3 \rightarrow R_2\\
		-R_1\rightarrow R_3 
	\end{array} &\xlrightarrow&
	\extaugmx{1&0&0\\0&1&0\\0&0&1}{2&-5&1\\-2&4&-1\\-1&2&0}
\end{align*}
Thus the inverse of the given coefficient matrix is
\[
\bmat{2&-5&1\\-2&4&-1\\-1&2&0}
\]
\subsection{} 
Multiply the inverse of the coefficient matrix with the given solutions to each linear equation to find the solution to the system:
\[
\bmat{2&-5&1\\-2&4&-1\\-1&2&0}\cdot \bmat{0\\-1\\2} = \bmat {7\\-6\\-2} \xlrightarrow \boxed{(7, -6, -2)}
\]

\section{}
\subsection{}
\[
	\extaugmx{1&1\\-2&-3\\1&2}{2\\-5\\3} \;
	\begin{array}{c}
		R_2 + 3R_1 \rightarrow R_2\\
		R_3 - R_1 \rightarrow R_3
	\end{array}\quad\xlrightarrow\qquad
	\boxed{\extaugmx{1&1\\1&0\\0&1}{2\\1\\1}}
\]
\[
\begin{array}{ccccccccc}
	A& & & & & & & &\\
	\bmat{1&1\\1&0\\0&1}&\cdot&\bmat{x\\y}&=&\bmat{1x+1y\\x\\y}&=&\bmat{2\\1\\1}&\rightarrow&\boxed{\begin{array}{c}x=1\\y=1\end{array}}
\end{array}
\]
\subsection{}
\[
\boxed{C = \bmat{0&1&0\\0&0&1}} \qquad \qquad
CA = \bmat{0&1&0\\0&0&1}\bmat{1&1\\1&0\\0&1} = \bmat{1&0\\0&1}
\]
\subsection{}
\[
\bmat{x\\y} = \bmat{0&1&0\\0&0&1}\bmat{2\\1\\1} = \bmat{1\\1} \begin{array}{cc}\rightarrow&x=1\\\rightarrow &y=1\end{array}
\]
\section{}
Find $x,y,z \in \reals$ such that at least one of $x,y,z \neq 0$ and
\[
\bmat{1&-2&4\\2&-4&9\\4&-8&18}\bmat{x\\y\\z} = \bmat{0\\0\\0}.
\]
Perform Gauss-Jordan Elimination on $A$ to get an $A'$:
\begin{align*}
	\begin{array}{c}
		R_2-2R_1 \rightarrow R_2\\
		R_3 -4R_1 \rightarrow R_3
	\end{array} & \xlrightarrow &
	\bmat{1&-2&4\\0&0&1\\0&0&2}&\\
	\begin{array}{c}
		R_1 - 4R_2 \rightarrow R_1\\
		R_3 - R_3 \rightarrow R_3
	\end{array} & \xlrightarrow &
	\bmat{1&-2&0\\0&0&1\\0&0&0}
\end{align*}
Then solve $A'x = 0$.
\[
	\bmat{1&-2&0\\0&0&1\\0&0&0}\bmat{x\\y\\z} = \bmat{x - 2y\\z\\0} = \bmat{0\\0\\0} \rightarrow
	\begin{array}{l}
		x - 2y = 0 \rightarrow \boxed{x = 2y}\\
		\boxed{z = 0}
	\end{array}
\]
Using these values, one non-trivial solution to $Ax = 0$ is \fbox{$(8, 4, 0)$}.
We can check this solution:
\[
\bmat{1&-2&4\\2&-4&9\\4&-8&18}\cdot \bmat{8\\4\\0} =
\begin{array}{rc}
	8-8+0 &=\\
	16-16+0 &=\\
	32-32+0 &=
\end{array}
\bmat{0\\0\\0}
\]

\end{document}