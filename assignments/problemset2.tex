\documentclass[]{article}
\usepackage{amsmath}
\usepackage{linalgjh} % useful for some things, I wanted the amatrix to use [] though.
\usepackage{multicol}

\renewcommand{\thesubsection}{\thesection.\alph{subsection}}

\newcommand{\mrow}[3]{\left[
	\begin{array}{ccc}
	#1 & #2 & #3
	\end{array}
	\right]}
\newcommand{\mcol}[3]{\left[
	\begin{array}{c}
	#1\\
	#2\\
	#3
	\end{array}
	\right]}

\newenvironment{augmx}[1]{%
	\left[\begin{array}{@{}*{#1}{c}|c@{}}
	}{%
	\end{array}\right]
}

\newcommand{\xlrightarrow}{\xrightarrow{\hspace{0.8cm}}}
\newcommand{\reals}{\mathbb{R}}

\begin{document}
	\begin{center}
		{\Large\textbf{MATH545: Problem Set 2}}\\
		\large Grey Cole\\
		February 06, 2026
	\end{center}
\section{}
\subsection{}
\begin{align*}
	\begin{augmx}{2}
		a&1&1\\2&a-1&1
	\end{augmx}
	\xlrightarrow
	\begin{linsys}{2}
		ax &+ &y &= &1\\
		2x &+ &(a-1)y &= &1
	\end{linsys}
	\xlrightarrow
	\begin{array}{c}
		y = 1-ax\\
		2x + (a-1)(1-ax) = 1
	\end{array}
\end{align*}
Solve for $x$ first, in terms of $a$:
\begin{align*}
	2x + a - 1 - a^2x + ax &= 1\\
	(-a^2+a+2)x+a&=2\\
	(2-a)(a+1)x&=2-a\\
	x&=\frac{1}{a+1}
\end{align*}
Then solve for $y$, in terms of $a$ with back substitution:
\begin{align*}
	y &= 1 - a(\frac{1}{a+1})\\
	y &= 1 - \frac{a}{a+1}
\end{align*}
The only value of $a$ that causes an invalid solution is $a=-1$. This non-solution is derived from the values of $x$ and $y$ calculated, but plugging it into the original system also causes a contradiction. So, the values of $a$ that make the linear system consistent are all real numbers except $-1$, i.e.
\[\boxed{
	a \in \reals \setminus \{-1\}.
}\]
\subsection{}
In most cases, when it is consistent the linear system has a unique solution. However, we can discover a case where the system has infinitely many solutions when we set the two equations equal to each other:
\begin{align*}
	ax + y &= 2x + (a - 1)y\\
	ax &= 2x + ay - 2y\\
	ax - ay &= 2x - 2y\\
	a(x - y) &= 2(x - y)\\
	a &= 2
\end{align*}
Plugging in $a=2$ to the system creates $0=0$, which means that there are an infinite number of solutions when $a = 2$.
\subsection{}
If we plug in $a = 0$,
\begin{align*}
	0x + y = 1 &\qquad& 2x + (0 - 1)y = 1\\
	\boxed{y = 1}	&\qquad&	2x - y = 1\\
	&\qquad&	2x - 1 = 1\\
	&\qquad&	2x = 1\\
	&\qquad&	\boxed{x = \frac{1}{2}}
\end{align*}
In particular, the solution to the system when $a=0$ is $(\frac{1}{2}, {1})$.

\section{}
\subsection{}
\begin{align*}
	&&
	\begin{bmatrix}
		0&2&1\\1&-3&-3\\1&2&-3
	\end{bmatrix}&\\
	\begin{array}{c}
		{R_1 + 2R_3 \rightarrow R_1}\\
		{R_2 - R_3 \rightarrow R_2}
	\end{array}
	&\xlrightarrow&
	\begin{bmatrix}
		2&6&-5\\0&-5&0\\1&2&-3
	\end{bmatrix}&\\
	{R_1-2R_3\rightarrow R_1} &\xlrightarrow&
	\begin{bmatrix}
		0&0&1\\0&-5&0\\1&2&-3
	\end{bmatrix}&\\
	\begin{array}{c}
		R_3 \leftrightarrow R_1\\
		-5R_2 \rightarrow R_2
	\end{array}
	&\xlrightarrow &
	\boxed{
	\begin{bmatrix}
		1&2&-3\\0&1&0\\0&0&1
	\end{bmatrix}
	}&
\end{align*}
\subsection{}
\begin{align*}
	&&
	\begin{bmatrix}
		-4&-2&-1\\-2&-3&0
	\end{bmatrix}&\\
	{R_1-2R_2\rightarrow R_1} &\xlrightarrow&
	\begin{bmatrix}
		0&4&-1\\-2&-3&0
	\end{bmatrix}&\\
	\begin{array}{c}
		-\frac{1}{2}R_2 \rightarrow R_1\\
		\frac{1}{4}R_1 \rightarrow R_2
	\end{array}
	&\xlrightarrow&
	\boxed{
	\begin{bmatrix}
		1&\frac{3}{2}&0\\
		0&1&-\frac{1}{4}
	\end{bmatrix}
	}&
\end{align*}

\section{}
\subsection{}
\begin{align*}
	\begin{bmatrix}
		3-2&2+1&0+3\\
		1+2&2-4&7+3\\
		1+0&6+0&0-1
	\end{bmatrix}
	\xlrightarrow
	\begin{bmatrix}
		1&3&3\\3&-2&10\\1&6&-1
	\end{bmatrix}
\end{align*}
\subsection{}
\begin{align*}
	\begin{bmatrix}
		-6+4&3+4&9+6\\
		-2-8&1-8&3-12-7\\
		-2+12&1+12&3+18
	\end{bmatrix}
	\xlrightarrow
	\begin{bmatrix}
		-2&7&15\\-10&-7&-16\\10&13&21
	\end{bmatrix}
\end{align*}
\subsection{}
\begin{align*}
	\begin{bmatrix}
		9+4&6-2&0-6\\
		3-4&-12-4&21-6\\
		3+0&18+0&0+2
	\end{bmatrix}
	\xlrightarrow
	\begin{bmatrix}
		13&4&-6\\-1&-16&15\\3&18&2
	\end{bmatrix}
\end{align*}
\subsection{}
\begin{align*}
	%&  & &  & &  & &  & &  & &  &\\
	&  &\begin{array}{c} 2AB\\
	\begin{bmatrix}
		-4&14&30\\-20&-14&-32\\20&26&42
	\end{bmatrix}
	\end{array}&
	&-&
	&\begin{array}{c} A^T\\
	\begin{bmatrix}
		3&1&1\\2&-4&6\\0&7&0
	\end{bmatrix}
	\end{array}&
	&=&
	&\begin{array}{c}\\
		\begin{bmatrix}
		-7&13&29\\-22&-10&-38\\20&19&42
	\end{bmatrix}\end{array}&  &
\end{align*}

\section{}
\[
	A_mA_k =
\begin{bmatrix}
	1-m & -m\\m & 1+m
\end{bmatrix}
\cdot
\begin{bmatrix}
	1-k&-k\\k&1+k
\end{bmatrix}
\]
\[
\hspace{-5em}
\begin{array}{rcl}
	\left[\begin{array}{cc}1-m & -m\end{array}\right]&\cdot
	\left[\begin{array}{c}
		1-k\\k
	\end{array}\right]
	&= (1-m)(1-k) - mk = 1-m-k+mk-mk=1-(m+k)
	\\
	\left[\begin{array}{cc}
		1-m&-m
	\end{array}\right]&\cdot
	\left[\begin{array}{c}
		-k\\1+k
	\end{array}\right]
	&= -k(1-m)-m(1+k)=-k+mk-m-mk=-(m+k)
	\\
	\left[\begin{array}{cc}
		m&1+m
	\end{array}\right]&\cdot
	\left[\begin{array}{c}
		1-k\\k
	\end{array}\right]
	&= m(1-k)+k(1+m)=m-mk+k+mk=(m+k)
	\\
	\left[\begin{array}{cc}
		m&1+m
	\end{array}\right]&\cdot
	\left[\begin{array}{c}
		-k\\1+k
	\end{array}\right]
	&= -km + (1+m)(1+k)=-km+1+m+k+mk=1+(m+k)
\end{array}
\]
\[
= \begin{bmatrix}
	1-(m+k) & -(m+k) \\ (m+k) & 1+(m+k)
\end{bmatrix}
= A_{m+k}
\]
Therefore $A_mA_k = A_{m+k}$.

\section{}
\[
\hspace{-5em}
\begin{array}{cccc}
\begin{array}{c}
	A = \begin{bmatrix}2&1\\1&1\end{bmatrix}\\
	\\
	M = \begin{bmatrix}a&b\\c&d\end{bmatrix}
\end{array}
\begin{array}{l}
	A \cdot M\\
	\left[\begin{array}{cc}2&1\end{array}\right]\cdot \left[\begin{array}{c}a\\c\end{array}\right] = 2a+c\\
	\left[\begin{array}{cc}2&1\end{array}\right]\cdot \left[\begin{array}{c}b\\d\end{array}\right] = 2b+d\\
	\left[\begin{array}{cc}1&1\end{array}\right]\cdot \left[\begin{array}{c}a\\c\end{array}\right] = a+c\\
	\left[\begin{array}{cc}1&1\end{array}\right]\cdot \left[\begin{array}{c}b\\d\end{array}\right] = b+d
\end{array}
\begin{array}{l}
	M \cdot A\\
	\left[\begin{array}{cc}a&b\end{array}\right]\cdot \left[\begin{array}{c}2\\1\end{array}\right] = 2a+b\\
	\left[\begin{array}{cc}a&b\end{array}\right]\cdot \left[\begin{array}{c}1\\1\end{array}\right] = a+b\\
	\left[\begin{array}{cc}c&d\end{array}\right]\cdot \left[\begin{array}{c}2\\1\end{array}\right] = 2c+d\\
	\left[\begin{array}{cc}c&d\end{array}\right]\cdot \left[\begin{array}{c}1\\1\end{array}\right] = c+d
\end{array}
\begin{array}{l}
	\\
	2a+c = 2a+b \rightarrow \boxed{c = b}\\
	\\
	2b+d = a + b \rightarrow \boxed{a = b + d}\\
	\\
	a+c = 2c+d \rightarrow \boxed{a = c + d}\\
	\\
	b+d = c + d \rightarrow \boxed{b = c}
\end{array}
\end{array}
\]
\\
Using these four equalities, we can simplify all four variables to just two:
\[
\hspace{-3em}
\begin{array}{cc}
	b = c & a = c + d
\end{array}
\Rightarrow
\begin{bmatrix}
	c+d & c\\c & d
\end{bmatrix}
\Rightarrow
\boxed{
M = \begin{bmatrix}
	n+k & n\\n & k
\end{bmatrix}
\text{for any } n, k \in \reals
}
\]

\end{document}