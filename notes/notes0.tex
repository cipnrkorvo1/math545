\documentclass{article}
\usepackage{graphicx} % Required for inserting images
%\usepackage[a4paper, total={6in, 8in}]{geometry}
\usepackage{amsmath}
\usepackage{amssymb}
\usepackage{amsthm}
\usepackage{empheq}
\usepackage{mathtools}

\title{Linear Algebra notes}
\author{Grey}

\newtheorem*{theorem}{Theorem}
\newtheorem{note}{Note}

\begin{document}
	\section{Systems of Equations}
	(1) $2x_1+3x_2=4$\\\
	(2) $x_1-x_2=7$\\
	\noindent A system is \underline{consistent} if it has a solution.\\
	Otherwise, the system is said to be \underline{inconsistent}.
	
	\subsection{Substitution}
	(2) $\Rightarrow x_1 = x_2 + 7$\\\
	(1) $\Rightarrow 4 = 2x_1 + 3x_2$\\
	$...$\\
	$5x_2 = -10 \Rightarrow x_2 = -2$\\
	
	\noindent $\therefore (1) \Rightarrow 7 = x_1 - x_2 = x_1 - (-2) = x_1 + 2$\\
	$\therefore x_1 = 5$\\
	$(5, -2)$
	
	\subsection{Elimination}
	$(2) -2x_1 - (-2)x_2=-2(7)$\\
	\\
	$(1) ~~~~2x_1 + 3x_2 = 4$\\
	$(2)-2x_1+2x_2=-14$\\
	$\Downarrow$\\
	$0x_1 + 5x_2 = -10$\\
	$x_2 = -2 \Rightarrow ...$
	
	\subsection{Equivalent Systems}
	\begin{align*}
		x_1 = a\\
		x_2 = b\\
		x_3 = c
	\end{align*}
	Consider
	\begin{empheq}[right=\empheqrbrace *]{align*}
		a_{11}&x_1+a_{12}x_2+...+a_{1n}x_n &= b_1\\
		a_{21}&x_1+a_{22}x_2+...+a_{2n}x_n &= b_2\\
		\vdots&  &\vdots\\
		a_{m1}&x_1+a_{m2}x_2+...+a_{mn}x_n &= b_m
	\end{empheq}
	This is a linear system of $m$ equations in $n$ unknowns, where $a_{ij}$ is the (numerical) coefficient of $x_j$ in equation $i$.
	\par\textbf{Definition:} Two linear systems of $m$ equations in $n$ unknowns are \underline{equivalent} if they have the same solution set.
	
	\begin{theorem}
		Let $(*)$ be the given linear system. Performing any of the following operations on $(*)$ will produce a system equivalent to $(*)$:
		\begin{enumerate}
			\item Interchange two of the equations.
			\item Multiplying any equation by a \underline{non-zero} constant.
			\item Adding a constant multiple of one equation to another.
		\end{enumerate}
		(These are "elementary" operations.)
	\end{theorem}
	
	\begin{align*}
		x_1+3x_2+5x_3=1\\
		3x_1-x_2+x_3=-3\\
		x_1+x_2+x_3=1
	\end{align*}
	Make the coefficient of $x_1$ in each subsequent equation 0 using the third elementary operation: 
	\begin{align*}
		x_1+3x_2+5x_3=1\\
		0x_1+10x_2-14x_3=-6\\
		0x_1-2x_2-4x_3=0
	\end{align*}
	Make the coefficient of $x_2$ in each subsequent equation 0 using the third elementary operation:
	\begin{align*}
		TODO
	\end{align*}
	
	\subsection{Representing a System as an Array: Matrices}
	\begin{align*}
		x_1+3x_2+5x_3=1\\
		3x_1-x_2+x_3=-3\\
		x_1+x_2+x_3=1
	\end{align*}
	can be represented as a matrix:
	\[\begin{bmatrix*}
		1 & 3 & 5 & 1\\
		3 & -1 & 1 & -3\\
		1 & 1 & 1 & 1
	\end{bmatrix*}\]
	An $m \times n$ matrix has $m$ rows and $n$ columns (the above is a $3 \times 4$ matrix). The above matrix is considered an \textit{"\underline{augmented} matrix for $(*)$"}.
	\textbf{TODO augmented definition}\\
	
	\noindent\textbf{Elementary operations on matrices:}\\
	\vspace{-1.5em}
	\textit{
	\begin{enumerate}
		\item Interchange two rows.
		\item Multiplying a row by a non-zero constant.
		\item Add to a row a constant multiple of another row.
	\end{enumerate}
	}
	
	\subsection{Gaussian elimination}
	To solve a system of equations, the desire is to get it in \textit{upper triangular form}. In other words, the goal is to transform its matrix into one of the form:
	\[\begin{bmatrix}
		* & * & * & *\\
		0 & * & * & *\\
		0 & 0 & * & *
	\end{bmatrix}\]
	
	\begin{note} \label{note}
		Solving becomes easier if the next coefficients after the required zeroes are then transformed 1:
		\[\begin{bmatrix}
			1 & * & * & *\\
			0 & 1 & * & *\\
			0 & 0 & 1 & *
		\end{bmatrix}\]
	\end{note}
	
	\noindent Example of solving the previous system using matrix elementary operations:
	
	\begin{multline*}
		\left[\begin{array}{ccc|c}
			1 & 3 & 5 & 1\\
			3 & -1 & 1 & -3\\
			1 & 1 & 1 & 1
		\end{array}\right]
		\xrightarrow{R_2 \leftarrow R_2 + (-3)R_1}
		\begin{bmatrix*}
			1 & 3 & 5 & 1\\
			0 & -10 & -14 & -6\\
			1 & 1 & 1 & 1
		\end{bmatrix*}\\
		\xrightarrow{R_3 \leftarrow R_3+(-1)R_1}
		\begin{bmatrix*}
			1 & 3 & 5 & 1\\
			0 & -10 & -14 & -6\\
			0 & -2 & -4 & 0
		\end{bmatrix*}\\
		\xrightarrow{R_3 \leftrightarrow R_2}
		\begin{bmatrix*}
			1 & 3 & 5 & 1\\
			0 & -2 & -4 & 0\\
			0 & -10 & -14 & -6
		\end{bmatrix*}\\
		\xrightarrow{R_2 \leftarrow (-\frac{1}{2})R_2}
		\begin{bmatrix*}
			1 & 3 & 5 & 1\\
			0 & 1 & 2 & 0\\
			0 & -10 & -14 & -6
		\end{bmatrix*}
		\\
		\xrightarrow{R_3 \leftarrow R_3 + (10)R_2}
		\left[\begin{array}{ccc|c}
			1 & 3 & 5 & 1\\
			0 & 1 & 2 & 0\\
			0 & 0 & 6 & -6
		\end{array}\right]\\
		\xrightarrow[(\text{Note } \ref{note})]{R_3 \leftarrow (\frac{1}{6})R_3}
		\left[\begin{array}{ccc|c}
			1 & 3 & 5 & 1\\
			0 & 1 & 2 & 0\\
			0 & 0 & 1 & -1
		\end{array}\right]\\
	\end{multline*}
	
	This technique for solving systems of linear equations is called \textbf{Gaussian elimination}.
	
	

	
	
\end{document}
